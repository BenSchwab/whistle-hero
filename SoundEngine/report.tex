%%%%%%%%%%%%%%%%%%%%%%%%%%%%%%%%%%%%%%%%%
% Large Colored Title Article
% LaTeX Template
% Version 1.1 (25/11/12)
%
% This template has been downloaded from:
% http://www.LaTeXTemplates.com
%
% Original author:
% Frits Wenneker (http://www.howtotex.com)
%
% License:
% CC BY-NC-SA 3.0 (http://creativecommons.org/licenses/by-nc-sa/3.0/)
%
%%%%%%%%%%%%%%%%%%%%%%%%%%%%%%%%%%%%%%%%%

%----------------------------------------------------------------------------------------
%  PACKAGES AND OTHER DOCUMENT CONFIGURATIONS
%----------------------------------------------------------------------------------------

\documentclass[DIV=calc, paper=a4, fontsize=11pt, twocolumn]{scrartcl}   % A4 paper and 11pt font size

\usepackage{lipsum} % Used for inserting dummy 'Lorem ipsum' text into the template
\usepackage[english]{babel} % English language/hyphenation
\usepackage[protrusion=true,expansion=true]{microtype} % Better typography
\usepackage{amsmath,amsfonts,amsthm} % Math packages
\usepackage[svgnames]{xcolor} % Enabling colors by their 'svgnames'
\usepackage[hang, small,labelfont=bf,up,textfont=it,up]{caption} % Custom captions under/above floats in tables or figures
\usepackage{booktabs} % Horizontal rules in tables
\usepackage{fix-cm}   % Custom font sizes - used for the initial letter in the document

\usepackage{sectsty} % Enables custom section titles
\allsectionsfont{\usefont{OT1}{phv}{b}{n}} % Change the font of all section commands

\usepackage{fancyhdr} % Needed to define custom headers/footers
\pagestyle{fancy} % Enables the custom headers/footers
\usepackage{lastpage} % Used to determine the number of pages in the document (for "Page X of Total")

% Headers - all currently empty
\lhead{}
\chead{}
\rhead{}

% Footers
\lfoot{}
\cfoot{}
\rfoot{\footnotesize Page \thepage\ of \pageref{LastPage}} % "Page 1 of 2"

\renewcommand{\headrulewidth}{0.0pt} % No header rule
\renewcommand{\footrulewidth}{0.4pt} % Thin footer rule

\usepackage{lettrine} % Package to accentuate the first letter of the text
\newcommand{\initial}[1]{ % Defines the command and style for the first letter
\lettrine[lines=3,lhang=0.3,nindent=0em]{
\color{DarkGoldenrod}
{\textsf{#1}}}{}}

%----------------------------------------------------------------------------------------
%  TITLE SECTION
%----------------------------------------------------------------------------------------

\usepackage{titling} % Allows custom title configuration

\newcommand{\HorRule}{\color{DarkGoldenrod} \rule{\linewidth}{1pt}} % Defines the gold horizontal rule around the title

\pretitle{\vspace{-30pt} \begin{flushleft} \HorRule \fontsize{30}{30} \usefont{OT1}{phv}{b}{n} \color{DarkRed} \selectfont} % Horizontal rule before the title

\title{Real-time Sound Feature Construction and Detection for Game Input } % Your article title

\posttitle{\par\end{flushleft}\vskip 0.5em} % Whitespace under the title

\preauthor{\begin{flushleft}\large \lineskip 0.5em \usefont{OT1}{phv}{b}{sl} \color{DarkRed}} % Author font configuration

\author{Ben Schwab, } % Your name

\postauthor{\footnotesize \usefont{OT1}{phv}{m}{sl} \color{Black} % Configuration for the institution name
Duke University, Math 361S Spring 2014 % Your institution

\par\end{flushleft}\HorRule} % Horizontal rule after the title

\date{} % Add a date here if you would like one to appear underneath the title block

%----------------------------------------------------------------------------------------

\begin{document}

\maketitle % Print the title

\thispagestyle{fancy} % Enabling the custom headers/footers for the first page

%----------------------------------------------------------------------------------------
%  ABSTRACT
%----------------------------------------------------------------------------------------

% The first character should be within \initial{}
\initial{T}\textbf{his paper investigates a system to process and identify a set of sound events (specifically whistles and snaps) for use as input to a video game. The environment of a game provides unique constraints that requires real time feature generation and recognition. The paper begins with a brief introduction of sound processing, a background on the Fast Fourier Transform - the key numerical method of this paper - followed by a description of the decision pipeline. }

%----------------------------------------------------------------------------------------
%  ARTICLE CONTENTS
%----------------------------------------------------------------------------------------

\section*{Introduction}

\par The last five years has marked enormous change in how people interact with video games. From the explosion of touch-based smart phone games to more complex technologies like Microsoft's Kinect, users are interacting with games in more natural and immersive manners. Sound based input has historically been a challenging problem because of the computational complexity of obtaining near $100\%$ accuracy required for an enjoyable video game experience. In this paper I propose a limited set of sound input actions: whistles and snaps. The small set allows faster and more accurate recognition than traditional speech based input. In addition, the choice of this input set allows a surprising amount of user control. As whistles have a distinct pitch it can be mapped to a two-dimensional scalable input (think of a joystick that can move only up and down). Snapping is a binary input that is most similar to pushing a button on a controller.
\par The primary focus of this paper is the construction of a JavaScript library which will generate ``whistle'' and ``snap'' events. However, in the conclusion, the paper discusses a proof of concept game. Currently the game can be accessed on \textbf{benschwab.github.io}.
\par



\lipsum[4] % Dummy text

%------------------------------------------------

\subsection*{Subsection 1}

\lipsum[5] % Dummy text

\begin{align}
A =
\begin{bmatrix}
A_{11} & A_{21} \\
A_{21} & A_{22}
\end{bmatrix}
\end{align}

\begin{itemize}
\item First item in a list
\item Second item in a list
\item Third item in a list
\end{itemize}

\lipsum[6] % Dummy text

%------------------------------------------------

\subsection*{Subsection 2}

\lipsum[7] % Dummy text

\begin{table}
\caption{Random table}
\centering
\begin{tabular}{llr}
\toprule
\multicolumn{2}{c}{Name} \\
\cmidrule(r){1-2}
First name & Last Name & Grade \\
\midrule
John & Doe & $7.5$ \\
Richard & Miles & $2$ \\
\bottomrule
\end{tabular}
\end{table}

%------------------------------------------------

\section*{Background}

\section*{Methods}

\section*{Results}

\section*{Discussion}

\section*{Conclusion}

\lipsum[8] % Dummy text

\begin{description}
\item[First] This is the first item
\item[Last] This is the last item
\end{description}

\lipsum[9] % Dummy text

%----------------------------------------------------------------------------------------
%  REFERENCE LIST
%----------------------------------------------------------------------------------------

\begin{thebibliography}{99} % Bibliography - this is intentionally simple in this template

\bibitem[Sauer, 2012]{Sauer:2006dg}
Sauer, Numerical Analysis 2nd ed. Pearson Education, Inc.
\newblock {\em The Fourier Transform}, p468--499.

\end{thebibliography}

%----------------------------------------------------------------------------------------

\end{document}